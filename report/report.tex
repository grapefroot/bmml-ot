\documentclass{article}


\usepackage{arxiv}

\usepackage[utf8]{inputenc} % allow utf-8 input
\usepackage[T1]{fontenc}    % use 8-bit T1 fonts
\usepackage{hyperref}       % hyperlinks
\usepackage{url}            % simple URL typesetting
\usepackage{booktabs}       % professional-quality tables
\usepackage{amsfonts}       % blackboard math symbols
\usepackage{nicefrac}       % compact symbols for 1/2, etc.
\usepackage{microtype}      % microtypography
\usepackage{lipsum}

\title{Project in BMML: Team with bayesian flavor}


\author{
   \And
  Dmitriy Salnikov   \\
  Skolkovo Institute of Science and Technology \\
  Moscow, Russia \\
   \And
  Aliaksandr Nekrashevich \\
  Skolkovo Institute of Science and Technology \\
  Moscow, Russia \\
   \And
  Nurlan Shagadatov\\
  Skolkovo Institute of Science and Technology \\
  Moscow, Russia \\
}

\begin{document}
\maketitle


% keywords can be removed
\keywords{GAN, deep learning, optimal transport, VAE, variational autoencoder, autoencoder, generative adversarial network}


\section{Introduction}

We study the paper \cite{main} 
exploring latent space modifications 
for generative models. The most known generative models 
are Generative Adversarial Networks (GANs) and Variational 
Auto Encoders (VAEs). The problem is that latent space operations
can create distributional mismatch, and the paper we study 
tries to resolve these mismatches with methods of 
optimal transport.

The rest of the report is organized as follows. In the 
Theoretical Explanation chapter we summarize the points presented
in the paper. In the Workflow description, we describe what 
cases and extensions we study. In the Experimental Results
chapter we present outcomes of our experiments.
Later, in Team contribution, we explain what is done by each team member. Finally, references are given.


\section{Theoretical Explanation}

The distributional mismatch is usually addressed in the following
steps. First, intuitive operator $y$ is constructed. Then resulting distribution is computed. After that, some cost is selected.
And we search for a minimal modification $\tilde{y}$ in terms of
cost, such that distribution is brought back to the prior.

These are examples with provided Gaussian Matched operation:

\begin{enumerate}
    \item 2-point interpolation. Here, $y = tz_1 + (1-t)z_2$. 
        To match the operation, result is divided by
        $\sqrt{t^2 + (1-t)^2}$.
    \item n-point interpolation. We weight 
        $y=\sum t_i z_i$, with sum of weights begin equal to 1.
        After that, divide by the euclidean norm of weights.
    \item Vicinity sampling. $y_j = z_1 + \varepsilon u_j$. 
        To match the operation, we divide by 
        $\sqrt{1 + \varepsilon^2}$.
    \item Analogies. The expression in terms of latent algebra
        looks like $z_3 + (z_2 - z_1)$. Since we work with
        triangle, in some sense we match with division on 
        $\sqrt{3}$.
\end{enumerate}

\section{Workflow Desription}

It was decided to separate the project into three parts:

\begin{enumerate}
    \item Reproducing the paper. This part is dedicated to GANs. 
        Here we train and adopt DCGAN,
        compute inception score, evaluate results on 
        different datasets.

    \item Applying distribution matching for VAE. Here, we work
        only with interpolation models.

    \item Correcting missing values for VAE. We take 
        initial image, corrupt it in different way, 
        and then reconstruct 
        it using optimal transport distribution matching.
\end{enumerate}

\section{Experimental Results}

TBD

\section{Team contribution}

\begin{enumerate}
        \item Dmitriy Salnikov: DCGAN experiments,
            distribution matching operations, presentation,
            team management, theoretical part of the paper
        \item Aliaksandr Nekrashevich: VAE experiments,
            inception score for DCGANs, report
        \item Nurlan Shagadatov: DCGAN experiments,
            missing values with VAE, presentation
\end{enumerate}

\bibliographystyle{unsrt}
\bibliography{references}

\end{document}
